\documentclass{article}
\usepackage{amsmath,
            algpseudocode,
            algorithm,
            listings,
            dirtree}

\setlength{\parindent}{0pt}

\begin{document}

Assuming that you have a directory tree of this format:\\

\dirtree{%
    .1 RtwoUtwoRuns.
    .2 CtwoPOfile.
    .2 RtwoUtwoOutputs.
    .3 outOne.
    .3 outTwo.
    .3 outEtc.
    .3 outN.
}

(Sorry about the weird naming. The latex package I used doesn't allow numbers and dots in the names for some reason...)

This directory is created from running R2U2 on the MontyCarlo simulation of a timeline. The CtwoPOfile (contracts.c2po) has all of the contract names for the timeline runs. The RtwoUtwoOutputs has all of the outputs of each run of the timeline through R2U2.





    \begin{algorithm}
        \caption{Creating Table Numbers from R2U2 Runs}\label{alg:cap}
        \begin{algorithmic}
            \State\noindent\textbf{Input: } .c2po, folder containing outputs from r2u2 as .txt
            \State\noindent\textbf{Output: } array containing aggregate data for multiple runs of r2u2
            \\
            \State agg\_data = empty\_array; \Comment{1}
            \While{contracts still in .c2po}
                \State agg\_data[n][0] = contract\_n \Comment{2}
            \EndWhile
            \\
            \While{file still in output folder}
                \For{line in .csv file}
                    \If{agg\_data[n][0] == contract\_name}
                    \State agg\_data[n][1] += 1 \Comment{3}
                    \EndIf

                    \If{agg\_data[n][0] == contract\_name \&\& failure}
                    \State agg\_data[n][2] += 1 \Comment{4}
                    \EndIf
                \EndFor    
            \EndWhile
            \\ 
            \For{row in agg\_data}
                \State agg\_data[n][3] = (agg\_data[2] $\div$ agg\_data[1]) $\times$ 100 \Comment{5}
                \State agg\_data[n][4] = (agg\_data[2] $\div$ total\_runs) $\times$ 100 \Comment{6}
            \EndFor
                \\ \\
            \Return return agg\_data

            
        \end{algorithmic}
    \end{algorithm}

    \newpage
    Comments:\\
    \begin{enumerate}
        \item Initalize an empty array for storing data
        \item Add contracts to first column of each row
        \item Add the number of times a contract appears to the second column of a contract
        \item If the line contains a failure for the contract name, add to third column
        \item Add the failure vs instance for a contract
        \item Add the failure vs total runs for a contract
    \end{enumerate}
    How it Works:\\
    The total number of runs is pulled from the total number of .csv's in the run folder.\\
    The aggregate data .csv that the gui table pulls from will initally be created by the .c2po that has all of the contract names and requirements\\
    If the .c2po file looks like this\\
    
    \begin{lstlisting}
    INPUT
        var_1: type;
    
    FTSPEC
        contract1_name: contract_1;
        contract2_name: contract_2;
        ...
        contractn_name: contract_n;
    \end{lstlisting}
    \vspace{1cm}
    Then the aggregate data .csv file will look like this\\

    \begin{table}[h!]
        \center
        \begin{tabular}{|l|l|l|l|l|}
            \hline
            contract\_1 & 0          & 0        & 0                   & 0                \\ \hline
            contract\_2 & 0          & 0        & 0                   & 0                \\ \hline
            ...         & 0          & 0        & 0                   & 0                \\ \hline
            contract\_n & 0          & 0        & 0                   & 0                \\ \hline
        \end{tabular}
        \caption{The contract names will create the .csv with the following four values set initally to 0}
        \label{table:1}
        \end{table}

    When a contract appears in an individual run of R2U2, the "occurances" number will update.
    When a contract fails in an individual run of R2U2, the "failures" number will update.

    Once all of the system runs have been processed by R2U2, the aggregate data .csv file will look like this\\
    
    \begin{table}[h!]
        \center
        \begin{tabular}{|l|l|l|l|l|}
        \hline
        contract\_1 & a          & b        & 0                   & 0                \\ \hline
        contract\_2 & c          & d        & 0                   & 0                \\ \hline
        ...         & ...        & ...      & 0                   & 0                \\ \hline
        contract\_n & e          & f        & 0                   & 0                \\ \hline
        \end{tabular}
        \caption{a,b,c,d,e,f are all positive integers}
    \end{table}

    The last two rows are filled in by iterating down each line in the aggregate data .csv
    \begin{table}[h!]
        \centering
        \begin{tabular}{|l|l|l|l|l|}
            \hline
            contract\_1 & a          & b        & (b $\div$ a) $\times$ 100 & (b $\div$ total\_runs) $\times$ 100 \\ \hline
            contract\_2 & c          & d        & (d $\div$ c) $\times$ 100 & (d $\div$ total\_runs) $\times$ 100 \\ \hline
            ...         & ...        & ...      & ...                                                 & ...                                                  \\ \hline
            contract\_n & e          & f        & (f $\div$ e) $\times$ 100 & (f $\div$ total\_runs) $\times$ 100 \\ \hline
        \end{tabular}
        \caption{a,b,c,d,e,f are all positive integers, total\_runs is the total number of runs for the system}
        \end{table}
\end{document}